\subsection*{IEEE 802.11ac - Wi-Fi 5}
The 5th generation WLAN is \ac{ac} and operates in the \SI{5}{\giga\hertz}  frequency range \cite{dhawankar_throughput_2018}.

According to \textcite{perahia_gigabit_2011}, \ac{ac} is a further evolution of IEEE 802.11n, where \ac{ac} adds to the known bandwidth of \text{IEEE 802.11n} of \SI{40}{\mega\hertz} the bandwidths \SI{80}{\mega\hertz},\SI{160}{\mega\hertz} and the interrupted bandwidth of \SI{80}{\mega\hertz} + \SI{80}{\mega\hertz}.

nach \textcite{sauter_wireless_2022} ist die Aufspaltung in zwei 80 Mhz Kanäle sehr nützlich, wenn das frequenzband reservierte Regionen enthält. Dadurch kann ein 160 Mhz breiter Kanal um  eine reservierte region des frequenzbandes gebaut werden.

The modulation technique used is \ac{OFDM}.
Additionally, a new \ac{MIMO} Downlink functionality for multiple users, called DL MU-MIMO, with up to 8 partial streams is introduced according to the authors. Together with the new \ac{MCS} from 64 \ac{QAM} to 256 \ac{QAM}, these three enhancements ensure that a higher data rate can be achieved. The maximum data rate is \SI{6.9}{\giga\hertz} according to the authors.

As declared by \textcite{abdelrahman_comparison_2015}, the 5th generation of WLAN has made it possible to expect better performance as in addition to a longer communication range compared to the previous IEEE 802.11 standards This statement could be proven at least for indoor range.  \textcite{dhawankar_throughput_2018} were able to demonstrate that \ac{ac} with a range of over \SI{60}{\metre} enables a longer indoor communication range than previous IEEE 802.11 standards.


new Physical Layer Very High Throughput (VHT) Physical Layer

80 Mhz

Beamforming 

\subsection*{IEEE 802.11ax - Wi-Fi 6}

The 6th generation of WLAN is \ac{ax}. \textcite{khorov_tutorial_2019} reveals what has changed from \ac{ac} to \ac{ax}. For this, the authors make the following statements.

\ac{ax} uses the same bandwidths in the \SI{5}{\giga\hertz} range and can also operate in the \SI{2.4}{\giga\hertz} frequency range with a maximum bandwidth of \SI{40}{\mega\hertz}. Similar to DL MU transmission, \ac{ax} enables UL MU transmissions. These can also use \ac{OFDMA} in addition to the already known \ac{MIMO} of \ac{ac}. \ac{OFDMA} groups the orthogonal frequency subcarriers into \ac{RU}s, which can be selected by the transmitter for optimal transmission to the receiver. This increases the \ac{SINR}.

An extension in the PHY layer are the new \ac{MCS}'s of up to 1024-\ac{QAM}. However, these should only be used with very good channel characteristics.
 For better outdoor communication \ac{ax} increases the \ac{OFDM} symbol duration from \SI{3.2}{\micro\second} for \ac{ac} to up to \SI{12.8}{\micro\second} and the \ac{OFDM} Guard Interval from a maximum of \SI{0.8}{\micro\second} for \ac{ac} to up to \SI{3.2}{\micro\second}.   
 
MIMO und OFDMA MU Streams

BSS Coloring

Backward Kompatibilität über CTS Reservierungen.

Tabelle Vergleich

\begin{table}[!ht]
	\centering
	\begin{tabular}{>{\raggedright}p{1.7cm}p{5.4cm}p{3.4cm}}
		\toprule
		Parameter & IEEE 802.11ac & IEEE 802.11ax \\
		\midrule
		Frequency bands & \SI{5}{\giga\hertz}&
		\SI{2.4}{\giga\hertz}, \SI{5}{\giga\hertz}, \SI{6}{\giga\hertz}\\
		Symbol Length & \SI{3.2}{\micro\second}&
		\SI{12.8}{\micro\second}\\
		\ac{OFDM} Subcarrier Spacing &
		\SI{312.5}{\kilo\hertz} &
		\SI{78.125}{\kilo\hertz} \\
		\ac{OFDM} Subcarriers in \SI{80}{\mega\hertz} &
		256 &
		1024 \\
		max. \ac{MCS} &
		256 -\ac{QAM} &
		1024 -\ac{QAM} \\
		max. \ac{GI} &
		\SI{0.8}{\micro\second} &
		\SI{3.2}{\micro\second} \\
		\bottomrule
	\end{tabular}
	\caption{Comparison of IEEE 802.11ac and IEEE 802.11ax}
	\label{tab:SensorNetworkApplications}
\end{table}