After I analyzed the impact on data rate and robustness of physical layer parameter, I will simulate a platooning
scenario in ns-3 to find the influence of the found physical layer configuration on the network performance.

Network performance metrics  Latency, update received etc.




\textcite{zhang_method_2009} defined a data frame of \SI{32}{\byte}, which includes an identifier, timestamp, Longitude, Latitude, Heading, Speed and Direction.
Diese Menge an Daten umfasst eine Grundmenge, welche für die Umsetzung eines Platooning Services ausreichen kann, wie die Autoren zeigen.
\textcite{schlingmann_aef_2019} spezifizieren die Datenmenge nicht weiter und weißen darauf hin, dass die benötigte Datenrate für Platooning Services gering ist.

Ich habe für die Simulation von Platooning Services die Datenmenge auf \SI{1}{\kilo\byte} gesetzt. Diese Datengröße ist eine Abstrahierung des Speicherplatzes, welcher möglicherweise für zusätzliche Daten oder Implementierungen von Authentifizierung -  und Sicherheitsmechanismen benötigt wird. Im Corn Harvest scenario können zusätzliche Daten z.B. der Füllstand der Transportmachine sein. 


Service Discovery

Rebroadcast by Count?
additional Traffic

MANET Service discovery

Visualisierung Netsimulyzer

Farbcodes

\begin{table}[H]
	\centering
	\begin{tabular}{>{\centering}p{2cm}p{4cm}p{4cm}}
		\toprule
		Harvest State & \ac{PD} & \ac{FH} speed\\
		\midrule
		H0 & \SI{30}{\tons\per\hectare}
        & \SI{10}{\kilo\metre\per\hour} \\
		H1 & \SI{40}{\tons\per\hectare}
        & \SI{8}{\kilo\metre\per\hour} \\
		H2 & \SI{50}{\tons\per\hectare}
        & \SI{6}{\kilo\metre\per\hour} \\
		\bottomrule
	\end{tabular}
	\caption{Frequency Channels numbers for \SI{2.4}{\giga\hertz} and \SI{5}{\giga\hertz} for the different \acf{BW}s of the IEEE 802.11 standard \cite{noauthor_ieee_2021-1}, which can be used for
	outdoor communication \cite{freq_plan_24G}, \cite{freq_plan_5G} and are configured in the Milesight Industrial Router UR75 for
	the field experiments.}
	\label{tab:markov_chain}
\end{table}

\begin{figure}
\centering
\begin{tikzpicture}[->, >=stealth', auto, semithick, node distance=3cm]
	\tikzstyle{every state}=[fill=white,draw=black,thick,text=black,scale=1]
    \node[state, circle, draw]    (A)               {H0};
	\node[state, circle, draw]    (B)[right of=A]   {H1};
	\node[state, circle, draw]    (C)[right of=B]   {H2};
	\path
	(A) edge[loop left]			node{0.7}	(A)
        edge[bend left,above]	node{0.3}	(B)
	(B) edge[bend left,below]	node{0.2}	(A)
        edge[bend left,above]   node{0.2}	(C)
        edge[loop]		        node{0.6}	(B)
	(C) edge[bend left,below]	node{0.3}	(B)
        edge[loop right]		node{0.7}	(C);
	%\node[above=0.5cm] (A){Patch G};
	%\draw[red] ($(D)+(-1.5,0)$) ellipse (2cm and 3.5cm)node[yshift=3cm]{Patch H};
	\end{tikzpicture}
\caption{Markov Chain for the states \acs{PD} 0, 1 and 2, which represent the current \acf{PD} and harvest speed from \autoref{tab:markov_chain}.}
\label{fig:your_label_here}
\end{figure}



\cite{aust_outdoor_2015} Link Metrics
