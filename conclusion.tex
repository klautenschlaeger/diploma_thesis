\chapter{Conclusion}
\acresetall

\begin{itemize}
\item summarize again what your paper did, but now emphasize more the results, and comparisons
\item write conclusions that can be drawn from the results found and the discussion presented in the paper
\item future work (be very brief, explain what, but not much how, do not speculate about results or impact)
\item recommended length: one page.
\end{itemize}
This thesis investigates the suitability of modern Wi-Fi for \ac{WIC}, which describes a wireless process data exchange between agricultural machines in the field.
The investigation is based on the prototypical application Agricultural Platooning Service in the corn harvest scenario.
Agricultural Platooning Services exchange process data via \ac{WIC} to guide a vehicle in a platoon with lateral and longitudinal
offset to a leading vehicle in the field.
In the corn harvest, Agricultural Platooning Services can be used to guide \ac{TM}s to the overloading positions of forage from \ac{FH}s .

I analyzed corn harvest process data, which indicated that the overloading position is usually at a distance below \SI{10}{\metre} at every angle relative to the \ac{FH}.

A field measurement campaign in the corn harvest scenario indicated that the signal strength of Wi-Fi suffers from shadowing effects due to the machine sizes.
In the open terrain of the agricultural domain, multipath effects occur easily and introduce additional interferences, which led to a \ac{PER} of \SI{30.0}{\percent}.
Additionally, I tested the Wi-Fi range in a farming environment and found that a communication range of \SI{820}{\meter}
for the \SI{5}{\giga\hertz} band and \SI{1.2}{\kilo\meter} for the \SI{2.4}{\giga\hertz} band
at a \ac{PER} of \SI{10.0}{\percent} is possible. \todo{Werte sind Wunschvorstellungen, die aber mit den Ergebnissen aus dem Feldtest nach deren Durchführung korregiert werden.}

I simulated different Wi-Fi physical layer configurations to analyze their impact on data rate and robustness.
IEEE 802.11ax introduces the new \ac{ER} mode, which enables a robust and long-range communication for service discovery transmission when used with \ac{DCM}.
The \ac{GI} of \SI{3.2}{\micro\second} in IEEE 802.11ax can be used to combat intersymbol interferences due to multipath effects in the agricultural environment.
The guidance data exchange mode can be configured to use low \ac{MCS} like \ac{HE}-\ac{MCS}3 with \ac{STBC} and \ac{LDPC} to achieve a robust communication,
which enables the \ac{WIC} to meet the required latencies and message intervals for Agricultural Platooning Services in the corn harvest scenario.

In my work, I was able to show that IEEE 802.11ax provides many configuration options to enable robust communication at the required data rate, range and latency.






\begin{comment}
    Untersuchen, welche Routing protocols
    Wi-Fi offers a wide range of physical layer configurations, which can be used to reduce the data rate and enable robust communication.

    Future work could investigate whether modern Wi-Fi can be used for other \ac{WIC} use cases.
    IEEE 802.11bd is a new standard, which is designed for \ac{WIC} in the industrial domain.
\end{comment}