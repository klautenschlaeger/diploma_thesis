\subsection*{Wi-Fi Physical Layer}
The further development of IEEE 802.11 is accompanied by a constant change of the physical layer. \textcite{sauter_wireless_2022} mentions, that all new enhancements of the physical layer of IEEE 802.11 are backward compatible to previous definitions of the it.

According to the Author, IEEE 802.11 initially used DSSS and FHSS as modulation methods.
Since IEEE 802.11g the modulation method \ac{OFDM} can be used in the \SI{2.4}{\giga\hertz} frequency band. the author explains \ac{OFDM} as following. \ac{OFDM} divides the transmission channel in subcarriers with different amplitudes, frequencies and phases. Each subcarrier is orthogonal to another one, as they send the information "Low", where only one other subcarrier is sending the information "High". 

Die Daten werden dann als sogenannte Symbole über die einzelnen Subcarrier gesendet. Der Abstand der zwischen den "High" der subcarrier wird als subcarrier spacing angegeben und entspricht der reziproken Symbollänge. Diese hat sich nun von \SI{3.2}{\micro\second} für IEEE 802.11n auf \SI{12.8}{\micro\second} für IEEE 802.11ax erhöht. Das entspricht einem subcarrier spacing von \SI{312,5}{\kilo\hertz} und \SI{78.125}{\kilo\hertz} respectively. 
Für die Standards IEEE 802.11p und IEEE 802.11bd gilt eine Symbollänge von \SI{6.4}{\micro\second}, was einem subcarrier spacing von \SI{156.25}{\kilo\hertz} entspricht \cite{jacobs}.

For the modulation and demodulation der übertragenden bits wird the FFT and IFFT are used respectively. Mit der Verringerung des subcarrier spacings entstehen mehr subcarriers im transmission channel, sodass die FFT Größe erhöht werden muss. 

\textcite{kauffels_wireless_2002} adds, that \ac{OFDM} can be used in the \SI{5}{\giga\hertz} frequency band since IEEE 802.11a. 

\subsubsection*{\acl{MCS}}

Um soviel bits wie möglich auf ein OFDM Symbol zu codieren können verschiedene \ac{MCS}s genutzt werden.   
These \ac{MCS}s for the IEEE 802.11 standards are based on \ac{PSK} or \ac{QAM} \cite{kauffels_wireless_2002}. 
Das kleinste \ac{MCS} ist Binary-\ac{PSK} und codiert \SI{1}{\bit} pro Symbol. IEEE 802.11ax hat das komplexeste \ac{MCS} von \num{256}-QAM IEEE 802.11ac auf \num{1024}-QAM und codiert damit jetzt \SI{10}{\bit} pro symbol \cite{axChallenge}.
In den \ac{V2X} Bereich kann ebenso e

An imaginary, theoretical transmission channel is usually specified as a square-wave signal in the frequency domain with the limits of both minimum and maximum amplitude and cut-off frequency. \textcite{kauffels_wireless_2002} defines the roll-off factor as a cosine-shaped flattening of the square signal between 0 and 1. In addition, the author points out that \ac{QAM} can generate high roll-off factors, so that signals interfere significantly more with adjacent channels.

In this regards the author recommends setting the parameters in an \ac{OFDM} system in such a way that first the coding rate and then the complexity of the \ac{MCS} is reduced in difficult transmission environments. Denn je mehr bits ein \ac{MCS} auf einem Symbol codiert, desto fehleranfällig ist die richtige Decodierung des Symbols.

\subsubsection*{\acl{FEC}}

Trotzdem können bei der übertragung bitfehler entstehen. Dazu erwähnt und erklärt \cite{knaufel} \ac{FEC} als eine Technique to reduce bit errors during transmission. \ac{FEC} adds redundant bits to the data. The receiver uses these redundant bits to check the integrity or correct errors of the received data. \todo{explain cr}

Dafür wird \ac{BCC} verpflichtend seit dem Standard IEEE 802.11n genutzt \cite{Challengeax} \cite{performanceLdpc}. \textcite{performanceLdpc} fügen dazu hinzu,dass es optional möglich ist \ac{LDPC} zu nutzen. Die Autoren führen dazu auf, dass \ac{LDPC} can achieve a better channel capacity performance. Das bestätigen auch \textcite{Challengeax} und weisen darauf hin, dass \ac{LDPC} auch higher computational cost erzeugt. 
IEEE 802.11ax Stations müssen \ac{LDPC} unterstützen, wenn Sie auf dem Standard IEEE 802.11ax unter den folgenden Bedingungen nutzen \cite{Challengeax} \cite{standardax}:
\begin{itemize}
	\item The used bandwidth is greater than \SI{20}{\mega\hertz}
	\item The chosen \ac{MCS} is \num{1024}-\ac{QAM}
	\item More then four transmission channels are used for the transmission. 
\end{itemize}
Damit erreicht IEEE 802.11ax \ac{CR} von  \nicefrac{1}{2}, \nicefrac{2}{3}, \nicefrac{3}{4}, and \nicefrac{5}{6} \cite{standardax}.
Ebenso nutzt IEEE 802.11p die Technik \ac{BCC}, welche beim Nachfolger IEEE 802.11ax von \ac{LDPC} abgelöst wurde \cite{jacobs} \cite{Comparisonpbd}. \textcite{Comparisonpbd} argue, that this step was important, as \ac{LDPC} offers better error correction possibilities for higher communication ranges greater than \SI{50}{\metre}.

Zusammen mit dem \ac{MCS} bilden die \ac{FEC} \ac{CR} ein Physical Layer specification. Diese wird nach den Standard benannt und umfasst die möglichen Ausprägungen für \ac{MCS} values and \ac{CR} des Standards. Für IEEE 802.11ax ergeben sich somit die HE-MCS index values in \autoref{tab:HEMCS}

\begin{table}[!ht]
	\centering
	\begin{tabular}{>{\raggedright}p{2cm}p{3cm}p{2cm}}
		\toprule
		HE-MCS index & \ac{MCS} & \ac{CR} \\
		\midrule
		\num{0} & Binary \ac{PSK}& \nicefrac{1}{2}\\
		1 &  Quadrature \ac{PSK}& \nicefrac{1}{2}\\
		2 &  Quadrature \ac{PSK}& \nicefrac{3}{4}\\
		3 &  \num{16}-\ac{QAM}& \nicefrac{1}{2}\\
		4 &  \num{16}-\ac{QAM}& \nicefrac{3}{4}\\
		5 &  \num{64}-\ac{QAM}& \nicefrac{2}{3}\\
		6 &  \num{64}-\ac{QAM}& \nicefrac{3}{4}\\
		7 &  \num{64}-\ac{QAM}& \nicefrac{5}{6}\\
		8 &  \num{256}-\ac{QAM}& \nicefrac{3}{4}\\
		9 &  \num{256}-\ac{QAM}& \nicefrac{5}{6}\\
		10 &  \num{1024}-\ac{QAM}& \nicefrac{3}{4}\\
		11 &  \num{1024}-\ac{QAM}& \nicefrac{5}{6}\\
		\bottomrule
	\end{tabular}
	\caption{HE-MCA index table nach \cite{standardax}}
	\label{tab:HEMCS}
\end{table}


\todo{Symbol length, GI, subcarrier spacing reciprocal}
\todo{Wellenausbreitung, Überlagerungseffekte, Reflexsion, Reflexsion nicht bei Metall}
\todo{Knauffel OFDM PHY}

\subsubsection*{\acl{GI}}
\textcite{pulimamidi_development_2007} erlären das Guard Interval als ein cyclic prefix OFDM Symbole vor Inter Symbol Interference und durch Inter Carrier Intereference. Inter Symbol Interference is caused by multipath delays , where the reflected delayed previous symbol can interfere with the current received symbol\cite{ravindranath_performance_2016}. Ebenso wird bei Inter Carrier Interference durch time-varying channel eine längere OFDM symbol duration erzeugt, dass genauso eine Interferenz with the following OFDM symbol entstehen kann \cite{van_duc_nguyen_intercarrier_2002}.

Über das Guard Interval führen \textcite{pulimamidi_development_2007} weiterhin folgendes auf. 
Da das Guard Interval die mögliche Interferenz auf das Folgesymbol verhindern soll, muss es mindestens so lang sein, sodass alle channel impulse responses mit dem entstehenden Delay im Guard Interval aufgefangen werden. 
\todo{longer GI WIfi 6 Outdoor Communication}
Das Guard Interval wird dann am Receiver wieder entfernt. Dadurch entsteht eine attentuation of bandwidth welche sich durch die folgende Formel beschreiben lässt: BERECHNET SICH ALS: NAHTLOSER ÜBERGANG \todo{Parameter einführen?}
\begin{equation}\label{eq:GI}
	\text{GI\_Bandwidth\_Attentuation} =
	\frac{
		\text{OFDM\_symbol\_duration} \times 100
	}{
		\text{OFDM\_symbol\_duration} + \text{GI}
	}
	,
\end{equation}
Seit IEEE 802.11n ist ein verkürztes \ac{GI} von  \SI{400}{\nano\second} nutzbar, welches gegenüber dem üblichen \ac{GI} von  \SI{800}{\nano\second} die  maximale Datenrate von \SI{270}{\mega\bit\per\second} auf \SI{300}{\mega\bit\per\second} steigert \cite{sauter_wireless_2022}.  IEEE 802.11ax unterstützt \ac{GI}s von \SI{800}{\nano\second}, \SI{800}{\nano\second} and \SI{800}{\nano\second} um einen besseren Schutz gegen \todo{better source} multipath effects in der indoor und outdoor communication zu ermöglichen. 
\subsection*{\acl{DCM}}
In order to introduce additional robostness \ac{DCM} can be applied to the physical layers of IEEE 802.11ax and IEEE 802.11bd according to \textcite{jacobs}. Die Autoren beschreiben \ac{DCM} als Möglichkeit Daten doppelt über zwei coherent carriers gesendet werden. Am Empfänger werden die Datenkopien mit dem log-likelihood ratio combiniert. Das erhöht die Empfangswahrscheinlichkeit der Daten auf Kosten der Datenrate. Die gleiche Anzahl von Daten benötigt nun doppelt so lange für die Übertragung. 


\todo{Frame stucture plotting maybe in tikz --> looks nicer}

\cite{standard} only for HE MCS 0, 1,2, 3, 4


Table 27-81—HE-MCSs for 242-tone RU, NSS = 3
Data MAximum 2  Stream only 1 and 2 spatial streams

Allowed relative constellation error versus constellation size

Table 27-51—Receiver minimum input level sensitivity

Table 27-52—Minimum required adjacent and nonadjacent channel rejection levels
optional feature \cite{standardax}

HE SU PPDU HE ER SU PPDU not for GI 800 ns \cite{standardax}

\subsubsection*{Extended Range}

\textcite{Deng} explains, that the HE ER SU \ac{PPDU} format is intendet to extend the range of a single station to access point transmission. Das wird nach den Autoren damit erreicht, dass die PPDU eine Wiederholung des HE-SIG-A Feldes beinhaltet.

\cite{Standard} spezifiziert, dass das HE ER SU PPDU format nur verwendet werden  darf, wenn 20 Mhz transmissions mit entweder 242 RU mit HE-MCS-0 - HE-MCS-2 oder 106 RU mit HE-MCS-0 auf einem spatial Stream verwendet werden.

Weiterhin wird nach \cite{Deng} über ein power-boosting der preamble, welches nach in \cite{standard} auf \SI{3}{\decibel} limitiert ist,  eine reliable transmission für weite Reichweiten gewährleistet. 



\subsubsection*{\ac{MIMO}}
In order to furhter exploit the physical layer capabilities, the single transmitting and receiving antenna systems called Single-Input-Single-Output can be extended to \ac{MIMO} - systems.
\textcite{sauter} describe the idea behind \ac{MIMO} as the usage of multiple transmit antennas and multiple receiving antenna. Dabei wird Spatial Multiplexing genutzt, sodass die gesendeten Signaler von jeder Antenne unterschiedlich an Objekten reflekiert werden und somit aus unterschiedlichen Richtung an den Receiver-Antennen empfangen werden können. 

Seit IEEE 802.11n sind bis zu vier MIMO-Streams möglich. Diese wurde bei IEEE 802.11ax nochmal auf bis zu acht MIMO-Streams erhöht. Da über jeden MIMO-Stream gleichzeitig Daten geschickt werden können, kann so die theoretische Datenrate in Abhängigkeit der nutzbaren Streams proportional steigen. Das ist nach \textcite{abbas} spatial multiplexing.

Jedoch sagen die Autoren zudem, dass MIMO dazu genutzt werden kann, um die Qualität des empfangenen Signals zu verstärken.  Das geschieht in Form von spatial diversity und \ac{STBC}.

MU-\ac{MIMO} no DCM possible

$NUM\_STS$ = Number of Spatial Streams
$n_{ss}$ = $NUM\_STS / (1 + STBC)$
\subsubsection*{\ac{STBC}}
Dabei werden den Autoren nach redundant copies of data transmitted via different antenna to the receiver. At the receiver the received data copies are combined and a maximum likelihood detector is applied in order to retain a high quality signal \cite{Santumon}.
\ac{STBC} is a technique used in Wi-Fi networks to improve the reliability and robustness of wireless communications. \ac{STBC} encodes multiple redundant copies of data at the transmit side, which are transmitted in different spatial streams to reduce the effects of fading and interference. At the receiver side, these multiple copies are combined to improve the signal quality and increase the data rate. This results in a more reliable and efficient wireless communication system, with improved data transfer speed and reduced error rates.
\ac{STBC} in Wi-Fi networks improve the reliability and robustness of wireless communications.
Here, \textcite{Stamoulis} has investigated the potential effect of \ac{STBC} on Wi-Fi. Their simulations showed that\ac{STBC} can increase the range and robustness for IEEE 802.11a. In addition, the authors concluded  that \ac{STBC} increases the \ac{SNR} in nearly all cases at the same throughput or even allows higher \ac{MCS} values to be used, thus allowing a higher throughput at the same \ac{SNR}. 


\cite{sauter} Reichweite und throughput higher wegen better SNR in Range 


26.11.9 STBC and DCM

HE Capabilities nur so gut, wie das schlechteste Glied

\cite{Standard} 607 not applicable with DCM

\cite{MAtthew gast 11n} \ac{STBC} is only support in one fifth of the Wi-Fi CERTIFIED n devices.
Group addressed frames
\subsubsection*{Bandwidth}