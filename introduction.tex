
\begin{itemize}
\item general motivation for your work, context and goals.
\item context: make sure to link where your work fits in
\item problem: gap in knowledge, too expensive, too slow, a deficiency, superseded technology
\item strategy: the way you will address the problem
\item recommended length: 1-2 pages.
\end{itemize}
With the growing demand for increased efficiency and autonomy in the agricultural domain, \ac{WIC} has emerged as a key technology to increase the automation
and digitalization of agricultural processes. \ac{WIC} describes a wireless connection between agricultural machines
in the field that enables the exchange of process data.

Given that there are many different agricultural technology companies worldwide, and a mix of their machines often cooperate in
the individual agricultural company, demand for interoperability between agricultural machines of different brands
has emerged.
In 2008, the \ac{AEF} was founded to develop and standardize this interoperability
\footnote{\url{https://www.aef-online.org/about-us/about-the-aef.html}}.

The AEF has defined a binary unit system, the ISO 11783 standard, for agricultural machinery communication, mainly tractors and
implements \cite{iglesias_enabling_2014}.
According to \textcite{schlingmann_aef_2019}, the ISO 11783 standard is known as
the ISOBUS system.

The authors mention that the AEF is currently working on other issues.
Among them is also \ac{WIC}.
The associated project group \ac{WIC} develops and standardizes solutions for \ac{M2M} wireless
communication between cooperating agricultural machines.

In order to implement \ac{WIC}, the \ac{WIC} project group has been searching for a suitable technology
that can realize the required data rates, latencies, robustness, and high transmission range.
The plans to implement \ac{WIC} are written down by members of the \ac{WIC} project group in \cite{schlingmann_challenges_2017}.
The authors consider the fundamental use of cellular networks as very problematic
because, according to \cite{itu2016facts}, only \SI{30}{\percent} of the land surface has network coverage.
For this reason, one major concern is that the required data cannot be exchanged because there is
no network connectivity around many fields.
Nevertheless, the authors want to leave the future \ac{WIC} system open
to cellular standards.


The authors' current focus is on Wi-Fi technologies, which must first be evaluated
for use in the agricultural environment.

In my final thesis, I indend to support the progress of the \ac{WIC} project group and investigate how modern Wi-Fi can enable \ac{WIC}. In particular, I will focus on the current Wi-Fi standard, IEEE 802.11ax.
During my research, I will study the use cases of \ac{WIC} in agriculture and the requirements and challenges for \ac{WIC} in agriculture.
In order to analyze the suitability of IEEE 802.11ax to enable \ac{WIC}, I will concentrate on the example use cases Agricultural Platooning Service and Video Streaming Service. Agricultural Platooning Service enables the exchange of process data
to guide a vehicle in a platoon with a lateral and longitudinal offset to a leading vehicle.
Video Streaming Services exchange video data to enable the driver to see the
remote content of screens and cameras of other vehicles in the field.

In the beginning, I will analyze agricultural process data to find the requirements and challenges of the mentioned use cases.
I will complement these with insights from past research on \ac{WIC} in agriculture and wireless technologies
in the agriculture domain in general.

The suitability of IEEE 802.11ax for \ac{WIC} depends on whether the required data rates, latencies, robustness, high
transmission ranges, and robustness in the harsh agricultural environment can be achieved.

By understanding past limitations of Wi-Fi technologies for outdoor communication networks
and exploring how IEEE 802.11ax addresses these limitations, I will investigate how IEEE 802.11ax can be configured.
For this purpose, I will first simulate how the capabilities of IEEE 802.11ax affect the data rate and the robustness of the wireless connection.
After learning suitable parameter settings for the physical and MAC layer of IEEE 802.11ax, I will simulate the use cases
Agricultural Platooning Service and Video Streaming Service to determine whether the requirements and challenges of the \ac{WIC} use cases
can be met by IEEE 802.11ax.


\todo{Add Field experiments to the thesis plan.}