\todo[color=blue]{Den Teil haben mehrere Leute Korrektur gelesen, sodass du den für den Verständnis auch nur überfliegen kannst.}
\begin{itemize}
\item general motivation for your work, context and goals.
\item context: make sure to link where your work fits in
\item problem: gap in knowledge, too expensive, too slow, a deficiency, superseded technology
\item strategy: the way you will address the problem
\item recommended length: 1-2 pages.
\end{itemize}
With the growing demand for increased efficiency and autonomy in the agricultural domain, \ac{WIC} has emerged as a key technology to improve the automation
and digitalization of farming processes. \ac{WIC} describes a wireless connection between agricultural machines
in the field that enables process data exchange.

Given that there are many agricultural technology companies worldwide, and a mix of their machines often cooperate in individual agricultural companies, demand for interoperability between agricultural machines of other brands
has emerged.
In 2008, the \ac{AEF} was founded to develop and standardize this interoperability
\footnote{\url{https://www.aef-online.org/about-us/about-the-aef.html}}.

The AEF is known fot its developed binary unit system, the ISO 11783 standard, for agricultural machinery communication, mainly tractors and
implements \cite{iglesias_enabling_2014}.
According to \textcite{schlingmann_aef_2019}, the ISO 11783 standard is named ISOBUS system.

The authors mention that the AEF is currently working on other issues.
Among them is also \ac{WIC}.
The associated project group \ac{WIC} develops and standardizes solutions for \ac{M2M} wireless
communication between cooperating agricultural machines.

In order to implement \ac{WIC}, the \ac{WIC} project group has been searching for a suitable technology
that can realize the required data rates, latencies, robustness, and high transmission range.
The plans to implement \ac{WIC} are written down by members of the \ac{WIC} project group in \cite{schlingmann_challenges_2017}.
The authors consider the fundamental use of cellular networks as very problematic
because, according to \cite{itu2016facts}, only \SI{30}{\percent} of the land surface has network coverage.
For this reason, one major concern is that the required data cannot be exchanged because there is
no network connectivity around many fields.
Nevertheless, the authors want to leave the future \ac{WIC} system open
to cellular standards.

The authors' current focus is on Wi-Fi technologies, especially IEEE 802.11p.

As of July 2021, the frequency spectrum of IEEE 802.11p in the United States of
America, ranging from \SIrange{5.85}{5.925}{\giga\hertz}, has been split.
The upper \SI{30}{\mega\hertz}
are reserved for Intelligent transportation systems now.
The lower \SI{45}{\mega\hertz} have
been released for unlicensed operations \cite{fcc_use_2021}. \todo[color=yellow]{andere Länder Japan, Korea?}

Since the use of IEEE 802.11p has now been partly limited by the  US Federal Communications Commission, the
WIC project group is looking for an alternative Wi-Fi technology that enables \ac{WIC}.

In my final thesis, I intend to support the progress of the \ac{WIC} project group and investigate how modern Wi-Fi can enable \ac{WIC}.
 In particular, I will focus on the current Wi-Fi standard, IEEE 802.11ax.
During my research, I will study the use cases of \ac{WIC} in agriculture and the requirements and challenges for \ac{WIC} in agriculture.
To analyze the suitability of IEEE 802.11ax to enable \ac{WIC}, I will concentrate on the example use case Agricultural Platooning Service.
Agricultural Platooning Service enables the exchange of process data
to guide a vehicle in a platoon with a lateral and longitudinal offset to a leading vehicle.

In the beginning, I will analyze agricultural process data to find the requirements and challenges of the mentioned use case.
I will complement these with insights from past research on \ac{WIC} in agriculture and wireless technologies
in the agriculture domain in general.

The suitability of IEEE 802.11ax for \ac{WIC} depends on whether the required data rates, latencies, robustness, high
transmission ranges and robustness in the harsh agricultural environment can be achieved.

By understanding past limitations of Wi-Fi technologies for outdoor communication networks
and exploring how IEEE 802.11ax addresses these limitations, I will investigate how IEEE 802.11ax can be configured.
For this purpose, I will first simulate how the capabilities of IEEE 802.11ax affect the data rate and the robustness of the wireless connection.
After learning suitable parameter settings for the physical and MAC layer of IEEE 802.11ax, I will simulate the use case
Agricultural Platooning Service to determine whether the requirements and challenges
can be met by IEEE 802.11ax.


\todo{Add Field experiments to the thesis plan.}