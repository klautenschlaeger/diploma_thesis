\autoref{fig:bearing_fh_tm} shows, that the \ac{TM} can be positioned at various distances and angles in relation to the \ac{FH}.
For one corn harvest scenario, \textcite{klingler_agriculture_2018} found out, that the \ac{RSS} can drop due to
shadowing effects caused by the size and shape of the \ac{FH} and the \ac{TM}.

In a field experiment, I want to analyze which positions of the \ac{TM} and \ac{FH} cause the shadowing effects, which
subsequently reduce the \ac{RSS} and how physical layer parameters like \ac{MCS} and \ac{STBC} can used to ensure a low
\ac{PER}.

For the experiment, I will use a \ac{CH} instead of a \ac{FH} as it has a similar shape and size as a \ac{FH} and is available.
The \ac{TM} will be a Tractor pulling a trailer of the type HW80.
Both machines will be equipped with a \ac{GPS} receiver and Wi-Fi devices which record the position, \ac{RSS} and the \ac{PER} of the
exchanged packets.
The \ac{CH} will be positioned in an agricultural field.
The tractor will start \SI{50}{\metre} behind the \ac{CH}, advance to the \ac{CH} and pass the \ac{CH} slowly with a
speed of \SIrange{1}{5}{\kilo\metre\per\hour} (\SIrange{0.28}{1.39}{\metre\per\second}) as shown in \autoref{fig:fieldDrive}.
\begin{figure}[]%
	\centering
	\includegraphics[width=0.2\textwidth]{figures/FieldExperimentDrive}
	\caption{Path around the static \acf{CH}, which the \acf{TM} will drive during the experiment to mimic various overloading positions}%
	\label{fig:fieldDrive}
\end{figure}
While driving along the specified path, the tractor will mimic various overloading positions, where shadowing effects can occur. After the tractor has passed the \ac{CH},
it will drive back to its starting position, and the experiment will be repeated with different overloading distances between the \ac{CH} and the tractor.

During the experiments, \ac{GPS} receivers at the agricultural machines will record the position and speed of the machines every \SI{1}{\second}.

The Wi-Fi setup consists of a Milesight Industrial Router UR75 \footnote{\url{https://iot-shop.de/en/shop/mil-ur75-500gl-g-p-w-milesight-ur75-500gl-g-p-w-industrial-cellular-5g-router-with-gps-wifi-and-poe-5677}}, which implements the standards IEEE 802.11 b/g/n in the \SI{2.4}{\giga\hertz} band and IEEE 802.11 a/n/ac in the \SI{5}{\giga\hertz} band and
IEEE 802.11 a/n/ac in the \SI{5}{\giga\hertz} band.
The router is equipped with two omnidirectional antennas for  \SI{2.4}{\giga\hertz} and \SI{5}{\giga\hertz} usage.
\textcite{brinkhoff_characterization_2017} and \textcite{paul_characterizing_2011}  already found out that placing the antenna higher above ground improves
the robustness and communication range of Wi-Fi networks in an outdoor environment.
As the regulation in the German Law StVZO §32 Abs. 2 limits the height of
every agricultural vehicle or combination of vehicles to less than \SI{4.0}{\metre}, the maximum antenna height is \SI{4}{\metre} above the ground. Therefore, I will mount the router on the tractor's roof at a height \SI{4}{\metre} above the ground.

I set up two Wi-Fi devices on the \ac{CH}, which are two UP Squared Board \footnote{\url{https://eu.mouser.com/datasheet/2/826/UP_Square_DatasheetV0_4-3084829.pdf}} with an Intel AX210 Wi-Fi module \footnote{\url{https://docs.alfa.com.tw/datasheets/alfa-network_ait-ax210-ex_latest.pdf}}.

Every Intel AX210 Wi-Fi module supports the IEEE 802.11ax standard for \SI{2.4}{\giga\hertz}, \SI{5}{\giga\hertz} and \SI{6}{\giga\hertz} band and is equipped with two antennas,
which support omnidirectional transmissions in the \SI{2.4}{\giga\hertz}, \SI{5}{\giga\hertz} and \SI{6}{\giga\hertz} band and have a gain of \SI{5}{\decibel}.
The boards are mounted on the roof of the \ac{CH} next to one another at a height \SI{4}{\metre} above the ground too.

The router on the tractor sets up a Wi-Fi \ac{AP}.
One of the boards on the \ac{CH} connects to the \ac{AP} of the router as a Wi-Fi \ac{STA} and hosts an iperf3 \footnote{\url{https://iperf.fr/}} server.
A notebook is connected via LAN to the router and runs an iperf3 client, which connects to the iperf3 server on the \ac{CH}.
The iperf3 client sends \SI{100}{\byte} UDP packets every \SI{100}{\milli\second} to the iperf3 server on the \ac{CH}.
The server records the received packets.

Many different Wi-Fi transmissions arise through the iperf3 UDP packets, the Wi-Fi manager of the Milesight Industrial Router, and the Intel AX210 Wi-Fi card.
These transmissions can be RTS/CTS, ACK, Data, Beacon or Probe request frame, displayed in \autoref{fig:fieldWifi}.
Through testing, I found out that the Wi-Fi manager of the Wi-Fi devices can apply VHT \ac{MCS} \numrange{0}{9} and \ac{STBC} as physical layer configurations.

\begin{figure}[H]%
	\centering
	\includegraphics[width=0.8\textwidth]{figures/FieldExperimentwifi}
	\caption{Wi-Fi transmissions between the Wi-Fi AP on the \acf{TM} and the Wi-Fi \ac{STA} on the \acf{CH}, which
	are recorded by a third Wi-Fi device in monitor mode on the \ac{CH}}
	\label{fig:fieldWifi}%
\end{figure}

The other UP Squared Board on the \ac{CH} uses the Wi-Fi card in the monitor mode and records every transmission in the \SI{5.6}{\giga\hertz} band using tcpdump \footnote{\url{https://www.tcpdump.org/}}.
Since the UP Squared board is placed next to the other board on the roof of the \ac{CH}, it can record the same signals the other board receives in the UDP transmission.
The tcpdump records are in pcap - format, which can be analyzed using Wireshark\footnote{\url{https://www.wireshark.org/}}.
Using Wireshark, I can identify possible retransmissions to calculate a \ac{PER}.
At the same time, the data contains the \ac{RSS} of each antenna and the physical layer parameters for every
transmission, allowing each transmission's robustness to be calculated as a function of the \ac{RSS} and the physical
layer configuration.

In order to get insights on the robustness of different physical layer \ac{BW}s and used frequencies, the frequency channels in \autoref{tab:fieldChannels} are configured in the
Every specified channel is used separately.
To be able to calculate the means and standard deviations of the result for every configuration, the experiment is repeated \num{5} times for each channel,
which means that the tractor drives \num{5} times the same path, which is displayed in \autoref{fig:fieldDrive}.

\begin{table}[H]
	\centering
	\begin{tabular}{>{\centering}p{2cm}p{4cm}p{4cm}}
		\toprule
		\ac{BW} & Channel number \SI{2.4}{\giga\hertz} & Channel number \SI{2.4}{\giga\hertz}\\
		\midrule
		\SI{20}{\mega\hertz} & \num{1}&
		\num{100} \\
		\SI{40}{\mega\hertz} &
		\num{3}
		& \num{102} \\
		\SI{80}{\mega\hertz} &
		- & \num{106} \\
		\bottomrule
	\end{tabular}
	\caption{Frequency Channels numbers for \SI{2.4}{\giga\hertz} and \SI{5}{\giga\hertz} for the different \acf{BW}s of the IEEE 802.11 standard \cite{noauthor_ieee_2021-1}, which can be used for
	outdoor communication \cite{freq_plan_24G}, \cite{freq_plan_5G} and are configured in the Milesight Industrial Router UR75 for
	the field experiments.}
	\label{tab:fieldChannels}
\end{table}

\subsubsection*{Trail Run}

\begin{figure}[H]%
	\centering
	\includegraphics[width=0.98\textwidth]{figures/trainRun}
	\caption{Position of the Wi-Fi Devices on the \ac{TM} and the \ac{CH} during the trail run}
	\label{fig:trailrunPositions}%
\end{figure}

\begin{figure}[H]%
	\centering
	\includegraphics[width=0.98\textwidth]{figures/wireless5}
	\caption{All QoS data transmissions between the Wi-Fi \acf{AP} on the \acf{TM} and the Wi-Fi \ac{STA} on the \acf{CH} in a trail run}
	\label{fig:trailrunAll}%
\end{figure}

\begin{figure}[H]%
	\centering
	\includegraphics[width=0.98\textwidth]{figures/wireless5_lan}
	\caption{All QoS data transmissions between the Wi-Fi \acf{AP} on the \acf{TM} and the Wi-Fi \ac{STA} on the \acf{CH} in a trail run,
	which were initiated by the iperf3 client on the \acf{TM}}
	\label{fig:trailrunIperf}%
\end{figure}

\begin{figure}[H]%
	\centering
	\includegraphics[width=0.98\textwidth]{figures/wireless5_ap}
	\caption{QoS data transmissions between the Wi-Fi \acf{AP} on the \acf{TM} and the Wi-Fi \ac{STA} on the \acf{CH} in a trail run,
	which were initiated by \ac{AP}}
	\label{fig:trailrunAP}%
\end{figure}

\begin{figure}[H]%
	\centering
	\includegraphics[width=0.98\textwidth]{figures/wireless5_lan}
	\caption{All QoS data transmissions between the Wi-Fi \acf{AP} on the \acf{TM} and the Wi-Fi \ac{STA} on the \acf{CH} in a trail run,
	which were initiated by the Wi-Fi \ac{STA} on the \ac{CH}}
	\label{fig:trailrunNode}%
\end{figure}

\begin{figure}[H]%
	\centering
	\includegraphics[width=0.8\textwidth]{figures/All_retries}
	\caption{Percentage of retransmissions for every transmission in the trail run}
	\label{fig:retriesCount}%
\end{figure}