\documentclass[]{nsm-thesis}
% options:
% [germanthesis] - Thesis is written in German
% [plainunnumbered] - Don't print numbers on plain pages
% [earlydraft] - Settings for quick draft printouts
% [watermark] - Print current time/date at bottom of each page
% [phdthesis] - switch to PhD thesis style
% [twoside] - double sided
% [cutmargins] - text body fills complete page

% Author name. Separate multiple authors with commas.

\usepackage{acro}



\author{Karl Christian Lautenschläger}
\birthday{29. Juni 1998}
\birthplace{Magdeburg}

% Title of your thesis.
\title{Suitability of Modern Wi-Fi for Wireless-Infield-Communication of Agricultural	Machines}

% Choose one of the following lines, changing the word "Computer Science" to match your degree program.
\thesistype{Diploma Thesis in Information Systems Engineering}\thesiscite{Diploma Thesis~(Diplomarbeit)}

\advisors{Christoph Sommer}
% List of advisors (without academic titles), separated by commas.

% List of referees (without academic titles), separated by commas.
\referees{Christoph Sommer, Burkhard Hensel}





\DeclareAcronym{WIC}{
	short=WIC,
	long=Wireless-Infield Communication,
}
\DeclareAcronym{M2M}{
	short=M2M,
	long=Machine-To-Machine,
}

\DeclareAcronym{MIMO}{
	short=MIMO,
	long=Multiple Input Multiple Output,
}
\DeclareAcronym{MCS}{
	short=MCS,
	long=Modulation
	and Coding Scheme,
}
\DeclareAcronym{MANET}{
	short=MANET,
	long=Mobile Ad Hoc Network
	and Coding Scheme,
}  

\DeclareAcronym{ASC}{
	short=ASC,
	long=Automatic Section Control,
}
\DeclareAcronym{m-ASC}{
	short=m-ASC,
	long=Meshed Automatic Section Control,
}
\DeclareAcronym{AEF}{
	short=AEF,
	long=Agricultural Industry Electronics Foundation,
}
\DeclareAcronym{FMIS}{
	short=FMIS,
	long=Farm Management Information System,
}
\DeclareAcronym{hTMM}{
	short=hTMM,
	long=Hybrid Threat Modeling Method,
}
\DeclareAcronym{STRIDE}{
	short=STRIDE,
	long= Spoofing\, Tampering\, Repudiation\, Information disclosure\, Denial of services\, and Escalation
	of privileges,
}
\DeclareAcronym{PASTA}{
	short=PASTA,
	long= Process for Attack Simulation and Threat Analysis,
}

\DeclareAcronym{ac}{
	short=802.11ac,
	long= IEEE 802.11ac,
}

\DeclareAcronym{ax}{
	short=802.11ax,
	long= IEEE 802.11ax,
}

\DeclareAcronym{OFDMA}{
	short=OFDMA,
	long= Orthogonal Frequency-Division Multiple Access,
}
\DeclareAcronym{OFDM}{
	short=OFDM,
	long= Orthogonal Frequency-Division Multiplexing,
}
\DeclareAcronym{RU}{
	short=RU,
	long= Resource Unit,
}
\DeclareAcronym{SINR}{
	short=SINR,
	long= Signal-to-Interference-plus-Noise Ratio,
}
\DeclareAcronym{QAM}{
	short=QAM,
	long=Quadrature Amplitude Modulation,
}
\DeclareAcronym{RSS}{
	short=RSS,
	long=Received Signal Strength,
}
\DeclareAcronym{PDR}{
	short=PDR,
	long=Packet Delivery Ratio,
}
\DeclareAcronym{LOS}{
	short=LOS,
	long=Line-of-sight,
}

% Define abbreviations used in the thesis here.
\acrodef{WSN}{Wireless Sensor Network}

\acrodef{ROI}{Region of Interest}{short-indefinite={an}, long-plural-form={Regions of Interest}}
\acrodef{ADAC}{German Automobile Association}{foreign={Allgemeiner Deutscher Automobilclub}}
\acrodef{CANhashing}[CAN]{Content Addressable Network}{extra={when referring to the distributed hash table}}
\acrodef{CANproto}[CAN]{Controller Area Network}{extra={when referring to the bus protocol}}

\begin{document}

\pagenumbering{roman}

\maketitle

\cleardoublepage

\TODO{This template is for use with \texttt{pdflatex} and \texttt{biber}. It has been tested with TeX~Live 2020 (as of 25 Oct 2020).}

\chapter*{Abstract}
\addcontentsline{toc}{chapter}{Abstract}
\begin{otherlanguage*}{american}

about 1/2 page:
\begin{enumerate}
    \item Motivation (Why do we care?)
    \item Problem statement (What problem are we trying to solve?)
    \item Approach (How did we go about it)
    \item Results (What's the answer?)
    \item Conclusion (What are the implications of the answer?)
\end{enumerate}

The abstract is a miniature version of the thesis.
It should be treated as an entirely separate document.
Do not assume that a reader who has access to an abstract will also have access to the thesis.
Do not assume that a reader who reads the thesis has read the abstract.

\end{otherlanguage*}


\chapter*{Kurzfassung}
\addcontentsline{toc}{chapter}{Kurzfassung}
\begin{otherlanguage*}{ngerman}

Gleicher Text (sinngemäß, nicht wörtlich) in Deutsch

\end{otherlanguage*}
\acresetall

\cleardoublepage
\tableofcontents
\TODO{The table of contents should fit on one page. When in doubt, adjust the \texttt{tocdepth} counter.}

\cleardoublepage
\pagenumbering{arabic}


\chapter{Introduction}
%\chapter{Einleitung}
\label{sec:introduction}

\begin{itemize}
\item general motivation for your work, context and goals.
\item context: make sure to link where your work fits in
\item problem: gap in knowledge, too expensive, too slow, a deficiency, superseded technology
\item strategy: the way you will address the problem
\item recommended length: 1-2 pages.
\end{itemize}

\section{\acl{WIC}}
\todo[color=blue]{Den Teil haben mehrere Leute Korrektur gelesen. Den musst du nicht überfliegen, da man ihn kaum für das Verständnis braucht.}
Since 2014, the \ac{WIC} project group has been working on the development of a for \ac{WIC} standard, which covers a standard for machine-to-machine communication, encryption, and security \footnote{https://www.aef-online.org/about-us/teams.html}.
\textcite{schlingmann_aef_2019} summarize the goals of the \ac{WIC} project team as follows:
\begin{itemize}
	\item Define use cases for \ac{WIC} in agriculture
	\item Evaluate the suitability of communication technologies
	\item Find suitable communication protocols
	\item Standardize the \ac{WIC} common software library
	\item Develop functional and security requirements and concepts
	\item Test first prototypes in regards of cross-brand comformance
	\item Write a application guideline
\end{itemize}
First steps have already been taken in this direction. The use cases and key scenarios are defined and explained by the authors as follows:
\begin{itemize}
	\item \textbf{Real-Time Machine-to-Machine Control} is the exchange of control data under real-time conditions with defined latency policies. This use case enables leader-follower scenarios where agricultural machines follow a leading agricultural machine at a lateral and longitudinal distance. Throughout this thesis, I will refer to Real-Time Machine-to-Machine Control as Agricultural Platooning Service.
	%\TODO{Fendt-Paper? Auch erwähnen eigentlich ja später, Beispiel aus AEF Paper?}
	\item \textbf{Streaming Services} are communications that stream video from remote cameras and monitors at a high data rate and low latency. The authors estimate the distance between the communication participants to be less than \SI{100}{\metre}. As a result, this data is available on another agricultural vehicle and can be analyzed and processed there.
	I will refer to Streaming Services as Agricultural Streaming Services in this thesis. %\TODO{Camera harvest streaming}
	\item \textbf{Process Data Exchange} describes the exchange of process data. One example is the exchange of already sprayed field areas to prevent multiple spraying of fertilizers and pesticides on the same field area by different machines. According to the authors, this \ac{WIC} use case requires long-range technologies because agricultural fields worldwide can be vast.
	%\TODO{ISO 11783-10 extended FMIS Data Interface}
	\item \textbf{Fleet Management \& Logistics} is the potential retrieval of data from the ongoing agricultural process. This information can influence economic or agronomic decisions of agricultural enterprises or service companies and is therefore required in a \ac{FMIS}.
	Since not all agricultural machines may be connected to the \ac{FMIS}, the \ac{WIC} project group is looking at how to use \ac{M2M} communications to bridge the missing communications infrastructure until the data reaches a machine that can connect to the \ac{FMIS}.
	\item \textbf{Road Safety} describes a use case which is already a project between the European Telecommunication Standard Institute and the \ac{AEF}. Since agricultural vehicles are repeatedly underestimated in their size and speed by other road users when they suddenly turn off the field onto the road, the other road users need to be warned in this situation. In this way, smart technologies in cars and motorcycles can brake these vehicles in advance and prevent possible accidents.
\end{itemize}

\todo[color=blue]{Wichtig:}
Considering that I investigate the Suitability of modern Wi-Fi for Wireless-Infield-Communication and modern Wi-Fi like IEEE 802.11ax is no long range technologies,
I will focus on investigating the suitability of IEEE 802.11ax for the \ac{WIC} use case Real-Time Machine-to-Machine Control.
Throughout this thesis, I will refer to real-time machine-to-machine control as Agricultural Platooning.
An example how farmers can benefit from Streaming Services or Agricultural Platooning Services is the corn harvesting and loading process.



\chapter{Fundamentals}
\label{sec:fundamentals}


\begin{itemize}
\item describe methods and techniques that build the basis of your work
\item include what's needed to understand your work (e.g., techniques, protocols, models, hardware, software, ...)
\item exclude what's not (e.g., anything you yourself did, anything your reader can be expected to know, ...)
\item review related work(!)
\item recommended length: approximately one third of the thesis.
\end{itemize}


\chapter{Field Measurements}


\chapter{Developed architecture / System design / Implementation / ...}


\begin{itemize}
\item describe everything you yourself did (as opposed to the fundamentals chapter, which explains what you built on)
\item start with a theoretical approach
\item describe the developed system/algorithm/method from a high-level point of view
\item go ahead in presenting your developments in more detail
\item recommended length: approximately one third of the thesis.
\end{itemize}

\chapter{Simulation}

\chapter{Evaluation}


\begin{itemize}
\item measurement setup / results / evaluation / discussion
\item whatever you have done, you must comment it, compare it to other systems, evaluate it
\item usually, adequate graphs help to show the benefits of your approach
\item each result/graph must not only be described, but also discussed (What's the reason for this peak? Why have you observed this effect? What does this tell about your architecture/system/implementation?)
\item recommended length: approximately one third of the thesis.
\end{itemize}



\chapter{Conclusion}


\begin{itemize}
\item summarize again what your paper did, but now emphasize more the results, and comparisons
\item write conclusions that can be drawn from the results found and the discussion presented in the paper
\item future work (be very brief, explain what, but not much how, do not speculate about results or impact)
\item recommended length: one page.
\end{itemize}



\cleardoublepage

\listofabbreviations
\clearpage

\listoffigures
\clearpage

\listoftables
\clearpage

\printbibliography

\end{document}
