\section{Data Rate}
\label{sec:DataRate}
\cite{dolinska_new_2019}, \cite{rochim_performance_2020} and
\cite{behara_performance_2022} used ns-3 to simulate the performance of IEEE 802.11ax in different scenarios.

As ns-3 is an open-source simulation software and was used by the mentioned other researchers in the past too, I used ns-3
to evaluate the effect of physical layer configuration on the achievable goodput between two nodes using IEEE 802.11ax Wi-Fi Netdevices
to exchange \ac{UDP} packets in ad-hoc Mode.

The setup consists of two nodes placed in static positions with a distance of \SI{20}{\metre}.
I chose the short communication range setup with no simulated interference to enable Wi-Fi transmission without any packets lost.

Every node is equipped with an IEEE 802.11ax Wi-Fi NetDevice, which is configured with the default parameters in
\autoref{tab:SimulationDataRate}.
\begin{table}[H]
   \centering
   \begin{tabular}{p{6cm}p{4cm}}
      General Parameters & \\
      \midrule
      Wi-Fi Standard & IEEE 802.11ax\\
      \ac{GI} & \SI{3200}{\nano\second}\\
      Frequency Spectrum & \SI{5.6}{\giga\hertz}\\
      \ac{BW} & \SI{20}{\mega\hertz}\\
      max. Transmission Power & \SI{25}{\decibel}\\
      Antenna Gain & \SI{5}{\dB}\\
      Spatial Streams & 2\\
      \bottomrule
   \end{tabular}
   \caption{Default simulation parameters for Wi-Fi Devices in the goodput simulations}
   \label{tab:SimulationDataRate}
\end{table}

A Constant Rate Wi-Fi Manager is used to set a constant data rate according to the fixed \ac{HE}-\ac{MCS} for data, non-uniform and control data transmissions.
The used frequency band is \SI{2.4}{\giga\hertz} or \SI{5}{\giga\hertz} as higher frequencies are less resistant to shadowing and fading, and a higher data rate is not needed for the \ac{WIC} use cases.
The Wi-Fi Netdevices operate in the frequency channels specified in \autoref{tab:fieldChannels}, which can be used for
outdoor Wi-Fi communication in Germany \cite{freq_plan_24G}, \cite{freq_plan_5G}.

The simulation is based on the ns-3 ConstantSpeedPropagationDelayModel and the  ns-3 FriisPropagationLossModel,
representing free-space signal propagation at the speed of light.
The FriisPropagationLossModel is suitable for outdoor scenarios with a clear \ac{LOS} between the nodes.
As the two nodes are only \SI{20}{\metre} apart, a transmission in the
in the \ac{LOS} is possible.

The ns-3 TableBasedErrorRateModel is used to model the physical layer error rate.
According to the ns-3 Wi-Fi design guide \footnote{\url{https://www.nsnam.org/docs/models/html/wifi-design.html\#default-table-based-error-model-validation}},
the TableBasedErrorRateModel derives the \ac{PER} from results of the Matlab WLAN Toolbox
regarding the physical layer parameters and \ac{SNR}.
The design guide recommends the TableBasedErrorRateModel for the IEEE 802.11ax standard.

As the Wi-Fi standard implements ACKs for every packet, every lost packet is repeated until it is received or the number of retries is
exceeded.
Platooning Services are time critical, so the number of retries should be as low as possible.
This is why additional retransmission mechanisms like TCP are not needed.
Therefore, the chosen transport layer protocol is UDP.

One node operates a \ac{UDP} server, and the other is a \ac{UDP} client.
The client sends \SI{1000}{\byte} \ac{UDP} packets to the server every \SI{0.1}{\micro\second}.
This packet interval
ensures that the packet queue of the client is never empty after starting the simulation.
The server receives the packets and sends an ACK back to the client.

For every physical layer configuration, the simulation runs five times for \SI{5}{\second}.
The goodput for every simulation run is calculated by dividing the number of received bytes at the \ac{UDP} Server by the simulation time.
The goodput is then averaged over all simulation runs per physical layer configuration, and the confidence interval with a confidence level of
\SI{95}{\percent} is calculated.

The theoretical data rate for the different physical layer configurations is retrieved from the function ns3::WifiMode::GetDataRate().

To verify the simulation software, I used different methods.
First, I confirmed that the theoretical data rate for the simulation's IEEE 802.11ax physical layer configurations equals the theoretical data rate specified in the IEEE 802.11ax standard \cite{ieee_standard_2021ax}.

I also used the MonitorSnifferRxCallback and the MonitorSnifferTxCallback of the ns-3 WifiPhy class to check the ongoing transmissions.
Both Callback functions can be added to WifiPhy objects of Wi-FiNetDevice and are called every time a packet is received or transmitted at the Wi-Fi Netdevice.
The function parameters are information about the packet, channel frequency and station ID and an instance of the WifiTxVector class.
The WifiTxVector instance in the current ns-3 version 3.37 describes all parameters of the transmission in accordance to the TXVECTOR field of the IEEE 802.11 standard \cite{ieee_standard_2021ax}. Additionally, the function parameters of the MonitorSnifferRxCallback contain the signal strength and
the noise power of the received packet.

Using the provided information from the MonitorSnifferRxCallback and the MonitorSnifferTxCallback, I was able to comprehend the ongoing transmissions and
verify the simulation results.

\subsubsection*{\acf{GI}}

In the first simulation, I varied the \ac{GI} of the Wi-Fi Netdevices for different \ac{HE}-\ac{MCS} values.
The results are shown in \autoref{fig:Data_rate_GI}.
The achieved goodput is plotted against the theoretical data rate for the different \ac{GI} values.
The theoretical data rate is always higher than the achieved goodput of the \ac{UDP} applications because Wi-Fi devices
cannot transmit data during other ongoing transmissions and the corresponding \acp{IFS}.
These transmissions can be \ac{ACK} and ad-hoc Beacon transmissions.
Due to this waiting time, the goodput is lower than the theoretical data rate.
\begin{figure}[H]
   \centering
   \includegraphics[width=0.69\textwidth]{figures/gi_dataRate_simulation.pdf}
   \caption{Achieved Goodput and theoretical Datarate of two Wi-Fi6 stations in Ad-Hoc Mode
   with \num{2} \acf{MIMO} streams and a bandwidth of \SI{80}{\mega\hertz} in regards to the chosen \acf{GI} and \ac{HE}-\ac{MCS} value}%
   \label{fig:Data_rate_GI}%
\end{figure}

The achieved goodput decreases as the \ac{GI} length increases.
This effect can be characterized by \autoref{eq:GI}.
The bandwidth attenuation for the possible \ac{GI} lengths is displayed in \autoref{tab:GIbandwidthAttenuation}.
The effect of the bandwidth attenuation for the different \ac{GI} lengths can be observed in the mean achieved goodput in
\autoref{tab:GIbandwidthAttenuation}, where the decrease of the mean goodput reflects the bandwidth attenuation of the decreasing \ac{GI} length.

A similar effect can be observed with higher \ac{HE}-\ac{MCS} values.
\begin{table}[H]
   \centering
   \begin{tabular}{>{\centering}p{3cm}p{3cm}p{3cm}}
      \toprule
      \ac{GI} & Mean achieved goodput & \ac{BW} attentuation\\
      \midrule
      \SI{800}{\nano\second} & \SI{31.15}{\giga\bit\per\second}&
      \SI{94}{\percent} \\
      \SI{1600}{\nano\second} &
      \SI{29.47}{\giga\bit\per\second}&
      \SI{89}{\percent} \\
      \SI{3200}{\nano\second} & \SI{26.52}{\giga\bit\per\second}&
      \SI{80}{\percent} \\
      \bottomrule
   \end{tabular}
   \caption{\acf{BW} attenuation and mean goodput for \ac{HE}-\ac{MCS}0 in regards to \acf{GI} length}
   \label{tab:GIbandwidthAttenuation}
\end{table}

\textcite{patil_ieee_2020} and \textcite{karmakar_s2-gi_2020} conducted similar simulations
for the \SI{400}{\nano\second} and \SI{800}{\nano\second} \ac{GI} lengths in IEEE 802.11n and IEEE 802.11ac, respectively.
Both papers state that shorter \ac{GI} lengths lead to higher goodput values under the condition that there is a short delay spread and a low
channel loss due to interference.

\subsubsection*{\acf{ER} Mode}

In the following simulation, I analyzed the effect of the \ac{ER} Mode on the goodput of the IEEE 802.11ax physical layer.
As mentioned in ns-3 Version 3.37, the \ac{ER} Mode is implemented as a \ac{HE} Capability with the new extended WifiPreamble.
But the new preamble in the \ac{HE} \ac{ER} SU \ac{PPDU} format is not used in ns-3 version 3.37.

As I was using the ConstantRateWifiManager, all parameters for the data transmission are set in the function
ConstantRateWifiManager::DoGetDataTxVector().
The function creates a WifiTxVector instance with the parameters of the
transmission.
There I overwrote the preamble type to the already implemented ns3 WifiPreamble::WIFI\_PREAMBLE\_HE\_ER\_SU, when
the \ac{ER} Mode is enabled and conditions for the \ac{ER} Mode in the IEEE 802.11ax standard \cite{ieee_standard_2021ax} are fulfilled.
Ns-3 version 3.37 implements the ns3::GlobalValue, which allows users to set global values for the simulation, which can be accessed
in every class without changing Constructor or function parameters.
This leaves the original functionality of the ns3 code intact.

I used the ns3::GlobalValue to create an instance named HE\_ER\_Mode, which is set to true at the start and read in the ns3::ConstantRateWifiManager::Do\-Get\-Data\-Tx\-Vector() function
to overwrite the preamble type.

Via the MonitorSnifferRxCallback and the MonitorSnifferTxCallback, I was able to verify that the ns3 WifiPreamble::WIFI\_PREAMBLE\_HE\_ER\_SU was used
for data transmission, when the following conditions were met: a) the \ac{ER} Mode is enabled, b) the number of spatial streams is \num{1}, c) the \ac{HE}-\ac{MCS} value is less than \num{3} and
d) the \ac{BW} is \SI{20}{\mega\hertz}.

The simulation results are shown in \autoref{fig:Data_rate_ER}, where the reduction of achieved goodput is plotted against the theoretical data rate for the different \ac{HE}-\ac{MCS} values.
The lost goodput is calculated by subtracting the achieved goodput when using the \ac{HE} \ac{ER} mode from the achieved goodput when using the normal \ac{HE} SU mode.
The only difference between \ac{HE} SU and \ac{HE} \ac{ER} SU transmissions is the preamble, which repeats the \ac{HE}-SIG-A field in the \ac{HE} \ac{ER} SU \ac{PPDU} format.
This results in a longer transmission time, reflecting the lower achieved goodput for the \ac{ER} Mode.
\begin{figure}[H]%
   \centering
   \includegraphics[width=0.69\textwidth]{figures/ER_dataRate_simulation.pdf}
   \caption{Lost Goodput between two Wi-Fi6 stations in Ad-Hoc Mode with a
   \acf{GI} of \SI{3200}{\nano\second} and a bandwidth of \SI{20}{\mega\hertz} in regards to \acf{HE}-\acf{MCS} when enabling the \ac{ER} mode}%
   \label{fig:Data_rate_ER}%
\end{figure}
The effect increases with smaller packet sizes because the longer preamble transmission time is more significant for smaller packets.
For higher \ac{HE}-\ac{MCS} values, more achievable goodput is lost because longer transmission time for the preamble could have
been used more \ac{OFDM} symbol transmissions, where more data is coded onto a symbol.

\subsubsection*{\acf{DCM}}
Using the \ac{DCM} in the IEEE 802.11ax physical layer also affects the achievable goodput.
As aforementioned, the \ac{DCM} is not supported by ns-3 version 3.37.
Therefore, I implemented the \ac{DCM} for this simulation in the ns-3 version 3.37 by transmitting a payload of twice the size, which represents the original payload and a
copy of it for the \ac{HE}-\ac{MCS} values 0, 1, 3 and 4, where \ac{MCS} is allowed.
Using \ac{DCM}, the receiver would apply maximum likelihood decoding to decode the original payload with a higher probability.

The results of the simulation are shown in \autoref{fig:Data_rate_DCM}, where the lost achieved goodput is plotted against
the theoretical data rate for the different \ac{HE}-\ac{MCS} values.
The theoretical data rate while using \ac{DCM} is half of the
theoretical data rate without \ac{DCM}, which complies with the IEEE 802.11ax standard \cite{ieee_standard_2021ax}.
\begin{figure}[H]%
   \centering
   \includegraphics[width=0.69\textwidth]{figures/DCM_dataRate_simulation.pdf}
   \caption{Achieved Goodput and theoretical Datarate of two Wi-Fi6 stations in Ad-Hoc Mode with for IEEE 802.11ax physical layer parameters of a \ac{GI} of \SI{3200}{\nano\second}, a \ac{BW} of \SI{20}{\mega\hertz} and \num{2} spatial streams  in regards to the number of the chosen \ac{HE}-\ac{MCS} value and whether \acf{DCM} is enabled}%
   \label{fig:Data_rate_DCM}%
\end{figure}

The achieved goodput is always lower than the theoretical data rate because data transmission time is lost to the header overhead and media access time.
Wi-Fi access is based on \ac{CSMACA}, meaning the stations have to wait for a random time before transmitting on a free channel.
Additionally, the channel can be occupied by ACK or ad-hoc beacon frames, which must be transmitted.

Using \ac{DCM} increases the ratio of achievable goodput to theoretical data rate because only one header and one ACK frame are transmitted per
\SI{2000}{\byte} payload, and the node has to go to the medium access procedure only once per \SI{2000}{\byte} payload.


\begin{comment}
   sommer
   header_fixed  medium access (beacon, ACK, RTS, CTS) the achievable goodput is always lower than the theoretical data rate.
   theoretical data rate
\end{comment}




\subsubsection*{\acf{STBC}}
\label{sec:STBCDataRate}
Another physical layer parameter that reduces the theoretical data rate for more robustness is the \ac{STBC}.
As mentioned, ns-3 version 3.37 does not support the \ac{STBC} for the IEEE 802.11ax standard.
Therefore, I reduced the number of
\ac{MIMO} streams from \num{2} to \num{1} when the \ac{STBC} is enabled. \ac{STBC} would transmit a redundant copy of the data on the second antenna, which would be combined
using the space-time block code (STBC) to increase the robustness and reliability of the transmission.
The results of the simulation are shown in \autoref{fig:Data_rate_STBC}, where the lost achieved goodput is plotted against
the theoretical data rate for the different \ac{HE}-\ac{MCS} values.
The theoretical data rate while using \ac{STBC} is half of the theoretical data rate without \ac{STBC},
which complies with the IEEE 802.11ax standard \cite{ieee_standard_2021ax}.
\begin{figure}[H]%
   \centering
   \includegraphics[width=0.69\textwidth]{figures/STBC_dataRate_simulation}
   \caption{Achieved Goodput and theoretical Datarate of two IEEE 802.11ax stations in Ad-Hoc Mode with for IEEE 802.11ax physical layer parameters of a \ac{GI} of \SI{3200}{\nano\second}, a \ac{BW} of \SI{20}{\mega\hertz} and 2 spatial streams  in regards to the number of the chosen \ac{HE}-\ac{MCS} value and whether \acf{STBC} is enabled}%
   \label{fig:Data_rate_STBC}%
\end{figure}

Other physical layer parameters, supported by the IEEE 802.11ax standard, are \ac{LDPC} and utilizing more \ac{MIMO} streams.
The effect that using more \ac{MIMO} streams for transmission increases the achievable goodput, is already known \cite[294-296]{sauter_wireless_2022, ieee_standard_2021ax, ieee_standard_2020}.
\ac{LDPC} adds no new effect for the achievable goodput, but more robustness which results in a lower \ac{PER}.

Using a different frequency spectrum, the achievable goodput can be limited due to the lower maximum \ac{BW} of \SI{40}{\mega\hertz} in the \SI{2.4}{\giga\hertz} frequency spectrum.
To analyze achievable goodput in the \SI{2.4}{\giga\hertz} frequency spectrum, I reran the simulations with the same configuration as in \autoref{sec:STBCDataRate}, but within the \SI{2.4}{\giga\hertz} frequency spectrum.
The results indicate that the achievable goodput is between \SIrange{0.1}{2.7}{\mega\bit\per\second} slower than in the \SI{5}{\giga\hertz} frequency spectrum.

Due to a signal extension of \SI{6}{\micro\second} in the \SI{2.4}{\giga\hertz} frequency spectrum, the transmission time is longer, which results in a lower achievable goodput.
\cite{ieee_standard_2009n} states that the signal extension in the \SI{2.4}{\giga\hertz} frequency spectrum enables
backwards compatibility with the older 802.11 standards, which need time to write their network allocation vector register.

The simulation results represent an ideal scenario where only two nodes exchange data.
In a real-world scenario in the agricultural domain, the number of nodes and the distance between the nodes may be higher.
The signal propagation will be influenced by the complex agricultural environment,
which can consist of large water areas, various obstacles and vegetation or weather conditions \cite{brinkhoff_characterization_2017}.
In case of the corn harvest scenario, \textcite{smolnik_5g_2020} points out that corn plants can be up to \SI{3}{\metre} high
and thus influence signal propagation.
These factors in the real-world scenario can lead to higher \ac{PER} and thus reduce the achievable goodput.

