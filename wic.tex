\begin{itemize}
	\item \textbf{Real-Time Machine-to-Machine Control}is the exchange of control data under real-time conditions with defined latency policies. This use case enables leader-follower scenarios where agricultural machines follow a leading agricultural machine at a lateral and longitudinal distance. 
	%\TODO{Fendt-Paper? Auch erwähnen eigentlich ja später, Beispiel aus AEF Paper?}
	\item \textbf{Streaming Services} are communications that stream video from remote cameras and monitors at a high data rate and low latency. As a result, this data is available on another agricultural vehicle and can be analyzed and processed there. The authors estimate the distance between the communication participants to be less than \SI{100}{\metre}. %\TODO{Camera harvest streaming}
	\item \textbf{Process Data Exchange} describes the exchange of process data. One example is the exchange of already sprayed field areas in order to prevent multiple spraying of fertilizers and pesticides on the same field area by different machines. According to the authors, long-range technologies must be used here because agricultural fields around the world can be very large.
	%\TODO{ISO 11783-10 extended FMIS Data Interface}
	\item \textbf{Fleet Management \& Logistics} is the potential retrieval of data from the ongoing agricultural process. This information can influence economic or agronomic decisions of agricultural enterprises or service companies and is therefore required in a \ac{FMIS}.
	Since not all agricultural machines may be connected to the \ac{FMIS}, the \ac{WIC} project group is looking at how to use \ac{M2M} communications to bridge the missing communications infrastructure until the data reaches a machine that can connect to the \ac{FMIS}.
	\item \textbf{Road Safety} describes a use case which is already a project between the European Telecommunication Standard 
	Institute and the \ac{AEF}. Since agricultural vehicles are repeatedly underestimated in their size and speed by other road users when they suddenly turn off the field onto the road, the other road users need to be warned in this situation. In this way, smart technologies in cars and motorcycles can be used to brake the vehicles in advance and prevent possible accidents.
\end{itemize}