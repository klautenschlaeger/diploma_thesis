Since 2014, the \ac{WIC} project group has been working on the development of a \ac{WIC} standard, which covers a standard for machine-to-machine communication, encryption, and security \footnote{https://www.aef-online.org/about-us/teams.html}.
\textcite{schlingmann_aef_2019} summarize the goals of the \ac{WIC} project team as follows:
\begin{itemize}
	\item Define use cases for \ac{WIC} in agriculture
	\item Evaluate the suitability of communication technologies
	\item Find suitable communication protocols
	\item Standardize the \ac{WIC} common software library
	\item Develop functional and security requirements and concepts
	\item Test first prototypes in regard to cross-brand comformance
	\item Write an application guideline
\end{itemize}
First steps have already been taken in this direction. The use cases and key scenarios are defined and explained by the authors as follows:
\begin{itemize}
	\item \textbf{Real-Time Machine-to-Machine Control} is the exchange of control data under real-time conditions with defined latency policies. This use case enables leader-follower scenarios where agricultural machines follow a leading agricultural machine at a lateral and longitudinal distance. Throughout this thesis, I will refer to Real-Time Machine-to-Machine Control as Agricultural Platooning Service.
	%\TODO{Fendt-Paper? Auch erwähnen eigentlich ja später, Beispiel aus AEF Paper?}
	\item \textbf{Streaming Services} are communications that stream video from remote cameras and monitors at a high data rate and low latency. The authors estimate the distance between the communication participants to be less than \SI{100}{\metre}. As a result, this data is available on another agricultural vehicle and can be analyzed and processed there.
	I will refer to Streaming Services as Agricultural Streaming Services in this thesis. %\TODO{Camera harvest streaming}
	\item \textbf{Process Data Exchange} describes the exchange of process data. One example is the exchange of already sprayed field areas to prevent multiple spraying of fertilizers and pesticides on the same field area by different machines. According to the authors, this \ac{WIC} use case requires long-range technologies because agricultural fields worldwide can be vast.
	%\TODO{ISO 11783-10 extended FMIS Data Interface}
	\item \textbf{Fleet Management \& Logistics} is the potential retrieval of data from the ongoing agricultural process. This information can influence economic or agronomic decisions of agricultural enterprises or service companies and is therefore required in a \ac{FMIS}.
	Since not all agricultural machines may be connected to the \ac{FMIS}, the \ac{WIC} project group is looking at how to use \ac{M2M} communications to bridge the missing communications infrastructure until the data reaches a machine that can connect to the \ac{FMIS}.
	\item \textbf{Road Safety} describes a use case which is already a project between the European Telecommunication Standard Institute and the \ac{AEF}. Since agricultural vehicles are repeatedly underestimated in their size and speed by other road users when they suddenly turn off the field onto the road, the other road users need to be warned in this situation. In this way, smart technologies in cars and motorcycles can brake these vehicles in advance and prevent possible accidents.
\end{itemize}

Considering that I investigate the suitability of modern Wi-Fi for Wireless-Infield-Communication and modern Wi-Fi like IEEE 802.11ax is no long range technology,
I will focus on investigating the suitability of IEEE 802.11ax for the \ac{WIC} use case Real-Time Machine-to-Machine Control.
Throughout this thesis, I will refer to real-time machine-to-machine control as Agricultural Platooning.
An example how farmers can benefit from Streaming Services or Agricultural Platooning Services is the corn harvesting and loading process.
