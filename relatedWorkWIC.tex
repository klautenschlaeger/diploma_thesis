

Since my undergraduate thesis \footnote{https://github.com/klautenschlaeger/mvsc} about \textit{
Wirelessly Networked Coordination of Automatic Section Control for Agricultural Machines}
, I have been working on the topic of \ac{WIC}. I conducted both field experiments and
simulations to investigate the performance of LoRa as a technology to exchange process data in
meshed Automatic Section Control, a prototypical application of connected vehicles in the agricultural domain.
A summary of my results is published in a paper \cite{lautenschlaeger_beyond_2022}.
In my undergraduate thesis and paper, I described the current state of research in the field of WIC.

The first research paper on WIC that I found was from \textcite{ali_multi-agent_2010}. The authors
developed a system based on General Packet Radio Service (GPRS) to exchange position data between \ac{TM}s and combine
harvesters to guide empty \ac{TM}s to a combine harvester.

\textcite{smolnik_5g_2020} describes the research project \textit{5G Netmobil} in which the authors investigated how existing technologies like
IEEE 802.11 or 3GPP LTE can be integrated into 5G technologies to enable Agricultural Platooning Services. The research plan was to evaluate the use of \ac{UDP} and Basic Transport Protocol (BTP) to exchange guidance data via
the underlying technologies 3GPP LTE and 5G V2X and IEEE 802.11p. The authors implemented a system using 802.11p,
as according to their technical analysis, this technology already fulfils the requirements for data rate, latency
and the number of participants. The authors report that the project results demonstrated that achieved
latencies were five times lower than the defined maximum latency of \SI{50}{\milli\second} for Agricultural Platooning Services.

Further research on \ac{WIC} is not based on cellular networks. \textcite{zhang_method_2009} used IEEE 802.15.4 to implement a
prototype of an Agricultural Platooning Service, where the developed system exchanges relevant control data between a leading tractor
to guide a following tractor.

\textcite{smolnik_5g_2020} states that the developed system of \textcite{zhang_method_2009} is part
of the project \textit{Elektronische Deichsel für landwirtschaftliche Arbeitsmaschinen (EDA)}
and it was further improved within the scope of project \textit{Elektronische Deichsel für landwirtschaftliche Arbeitsmaschinen mit Umfeldsensorik und zusätzlichen Geoinformationen (EDAUG)}.


\textcite{klingler_agriculture_2018} investigated how IEEE 802.11p can be used for \ac{WIC}. Experiments revealed
that data could be exchanged over a maximum range of 1700 m, where \ac{LOS} was lost. But during the
measurement in an agricultural work scenario from the corn harvest, there were collapses in the \ac{RSS}
due to shadowing effects of the machines. The authors point out that the size and shape of the forage harvester
can cause intensified shadowing effects.

There are also more developments from the industry in the field of \ac{WIC}. In this context, \textcite{thomasson_review_2018} describe the John Deere Machine Sync and Case IH V2V systems as follows:

John Deere Machine Sync enables the \ac{WIC} use cases Process Data Exchange and Agricultural Platooning Service. \textcite{liu_automation_2022} have extended the system to use Combine Harvesters, adding that the Machine Sync system is based on \textcite{metzler_system_2006}'s patent.
\textcite{smolnik_5g_2020} adds that John Deere Machine Sync is only available for a subgroup of John Deere machine types and cannot be used with machines of other brands.

Case IH V2V also offers an agricultural platooning service. However, according to the authors, the system can only be used for harvesting and loading scenarios.

Also currently on the market is the Raven Autonomy™ Driver Assist Harvest Solution \footnote{https://ravenind.com/products/autonomy/driver-assist-harvest-solution} system from Raven Industries. This system allows the harvester to take control of a \ac{TM} from a distance of \SI{70}{\metre}. The harvester then automatically guides the \ac{TM} into the perfect position to load the harvested crop onto the \ac{TM} via the spout. Once the harvesting and loading process is complete, the driver of the \ac{TM} driver retakes control.

A comparable system is CartACE from AgLeader \footnote{https://www.agleader.com/harvest/cartace/}

The technology used in the mentioned systems is not known. In response to questions about how the systems can be used on farms worldwide and what prerequisites must be created for this, the manufacturers refer to the regional distribution options.

Wireless communication technologies are also used to implement wireless sensor networks in the agricultural domain.

According to \textcite{ahmed_internet_2018}, wireless sensor networks in the agricultural domain can be used to monitor soil and water conditions, plant diseases, and farm automation solution
or track animals or assets. The authors mention similar requirements for wireless sensor network applications compared to \ac{WIC} applications.
For example, asset-tracking applications require low latency and must support asset mobility.
The authors' results indicate that fog computing can lessen the latency and the required bandwidth compared to cloud computing.
When a higher data rate is required, the authors recommend Wi-Fi technologies like IEEE 802.11n or IEEE 802.11ac.

As wireless sensor networks for agricultural applications, they must be able to operate in the same agricultural environment as \ac{WIC} applications.
\textcite{brinkhoff_characterization_2017} describe that they expect a limited cellular network coverage and complex outdoor
environments with large water areas, different crop vegetation, and other obstacles or various weather conditions. The
researchers developed a wireless sensor network based on IEEE 802.11b, where they exchanged data between an \ac{AP} and multiple stations on
a cotton and rice field. The authors report that they easily achieved a communication range of \SI{1000}{\metre} in a \ac{LOS} scenario.
They mention that different wheater conditions have little impact on communication reliability. A significant influence on the communication range is the height above ground or the crop vegetation, where the authors recommend using at least a height of \SI{0.2}{\metre}.


Wi-Fi technologies are also used in various outdoor scenarios.
The outdoor performance of Wi-Fi technologies in different use cases and scenarios have been investigated by various researchers.
\textcite{aust_outdoor_2015} surveyed past research on outdoor performance of IEEE 802.11 technologies. The authors summarize results of
urban, rural, desert or water surface scenarios with different environmental conditions and antenna configurations. I focus on the
findings for Wi-Fi with omnidirectional antennas in rural areas.

In their general findings, they cite \cite{sheth_packet_2007} and \cite{aguayo_link-level_2004}, which name the intersymbol
interference due to multipath effects as main reason for packet losses in Wi-Fi outdoor communication. The authors conclude,
that using directional antennas instead of omnidirectional antennas can lead to fewer multipath effects. Furthermore, the authors add
that external Wi-Fi interference is only expected in urban areas and is not very likely in rural areas.

Due to the experiment results of \cite{chebrolu_long-distance_2006}, the authors state that the weather conditions only have a
small impact of \SIrange{1}{2}{\decibel} on the outdoor communication. These findings are also confirmed by \cite{brinkhoff_characterization_2017}
in their experiments on a rice and cotton field.

\textcite{paul_characterizing_2011} analyzed the open outdoor performance for different physical layer and MAC layer configurations of
IEEE 802.11n. The authors conducted experiments with different omnidirectional antenna constellations and spacings ranging from
\SIrange{0}{25.4}{\centi\metre}. They found out that no positive impact on the communication can be achieved by
changing antenna constellations or spacings.

For the MAC layer with its \ac{CSMACA} mechanism,\textcite{aust_outdoor_2015} state that the increased propagation delay in outdoor scenarios
can cause a higher number of packet collisions. They refer to to findings of \cite{paul_characterizing_2011} and recommend
using Block Acknowledgement and Frame Aggregation. These mechanisms introduce a medium access control frame overhead, which cost less transmission time than
retransmission of a whole frame or a single acknowledgment frame.

\todo{landnetz}

