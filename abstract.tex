\chapter*{Abstract}
\addcontentsline{toc}{chapter}{Abstract}
\begin{otherlanguage*}{american}

\begin{comment}
    about 1/2 page:
\begin{enumerate}
    \item Motivation (Why do we care?)
    \item Problem statement (What problem are we trying to solve?)
    \item Approach (How did we go about it)
    \item Results (What's the answer?)
    \item Conclusion (What are the implications of the answer?)
\end{enumerate}

The abstract is a miniature version of the thesis.
It should be treated as an entirely separate document.
Do not assume that a reader who has access to an abstract will also have access to the thesis.
Do not assume that a reader who reads the thesis has read the abstract.

\end{comment}

\TODO[color=yellow]{Ihre Verbesserungsvorschläge?}

\ac{WIC} describes a wireless data exchange between agricultural machines in the field.
A prototypical use case is an Agricultural Platooning Service, which exchanges guidance data positions following
vehicles in a platoon with a lateral and longitudinal offset to a leading vehicle in the field.
This thesis investigates the suitability of modern Wi-Fi for \ac{WIC} based on the use case Agricultural Platooning Service.
I analyzed process data and derived requirements of Agricultural Platooning Services in the the corn harvest scenario.

Field measurements in an agricultural environment indicated that Wi-Fi communication can range up to \SI{2500}{\metre} in
a line-of-sight scenario.
However, signal strength can suffer from multipath, shadowing and fading effects due to the
agricultural environment and the machine sizes.


I simulated different Wi-Fi physical layer configurations that reduce the data rate but
enable robust communication to overcome these challenges.
The results enable configuration for long-range service discovery and uniform guidance data exchange modes
in the Agricultural Platooning Service.
Finally, it was shown that the required latencies and message intervals for Agricultural Platooning Services could be met in the corn harvest
scenario.

\end{otherlanguage*}

\chapter*{Kurzfassung}
\addcontentsline{toc}{chapter}{Kurzfassung}
\begin{otherlanguage*}{ngerman}


\acf{WIC} beschreibt den drahtlosen Datenaustausch zwischen landwirtschaftlichen Maschinen auf dem Feld.
Ein prototypischer Anwendungsfall sind elektronische Deichseln in der Landwirtschaft, die Positionierungsdaten austauschen, um nachfolgender
Fahrzeuge eines Zuges mit einem latitudinalen und longitudinalen Versatz zu einem führenden Fahrzeug im Feld zu führen.
In dieser Arbeit wurde die Eignung von modernem Wi-Fi für \ac{WIC} anhand dieses Anwendungsfalls untersucht.
Ich analysierte Prozessdaten und leitete daraus Anforderungen für die Anwendung von elektronischen Deichseln in der Maisernte ab.

Feldmessungen in einer landwirtschaftlichen Umgebung ergaben, dass die Wi-Fi-Kommunikation bei Sichtverbindung eine Reichweite von bis zu \SI{2500}{\metre}
möglich ist.
Die Signalstärke kann jedoch aufgrund der landwirtschaftlichen Umgebung und der Größe der Maschinen unter Mehrwegeffekten, Abschattungen und Pfadverlusten leiden.
landwirtschaftlichen Umgebung und der Größe der Maschinen leiden.

Ich habe verschiedene Wi-Fi Konfigurationen der physikalischen Schicht simuliert, die die Datenrate reduzieren, aber
robuste Kommunikation ermöglichen, um diesen Herausforderungen zu begegnen.
Die Ergebnisse ermöglichen eine Konfiguration für den Broadcast des Dienstes über große Entfernungen und den uniformen Austausch von
Positionierungsdaten für die elektronische Deichsel in der Landwirtschaft.
Schließlich konnte gezeigt werden, dass die erforderlichen Latenzen und Nachrichtenintervalle für landwirtschaftliche Platooning-Dienste in einem Maisernte
Szenario eingehalten werden können.

\end{otherlanguage*}
\acresetall



\cleardoublepage