\section{Network Simulation}
\label{sec:network_simulation}

Network simulations are considered the main evaluation method for IEEE 802.11ax networks by the IEEE 802.11ax task group \cite{omar_survey_2016}.
According to \textcite{manpreet_survey_2014} network simulation can analyze the performance of network applications and routing protocols in various scenarios.
\textcite{kumar_simulators_2012} argue that network simulation is a cost-effective and flexible method to evaluate the performance of a network protocol.

The authors define the following \num{4} steps to build a network simulation:
\begin{itemize}
    \item Simulation model, which defines the network components and their behaviour
    \item Simulation scenario, which describes the network topology and the traffic model
    \item Data collection of the simulation results
    \item Analysis, verification and visualization of the simulation results
\end{itemize}

Various tools are already available, such as OPNet, ns-2 and ns-3, OMNeT++, Qualnet \cite{kumar_simulators_2012, manpreet_survey_2014} or Matlab \cite{keller_simulation_2021}.
\textcite{kumar_simulators_2012} compares different available simulators for the simulation of wireless networks and concludes that,
in general, ns-2, ns-3 and OMNeT++ are the most popular simulators for academic research of wireless networks.
\textcite{keller_simulation_2021} mentions that ns-3 keeps a more accurate Implementation of the IEEE 802.11 standard than OMNeT++.
So far, I know that OMNeT++ supports some IEEE 802.11ac modes via the library INET Release 4.0.0 \footnote{\url{https://inet.omnetpp.org/2018-06-28-INET-4.0.0-released.html}}.
Consequently, OMNeT++ seems unsuitable for IEEE 802.11ax simulations and I looked at ns-3.

Ns-3 is well documented in the ns-3 manual in the \textit{./doc} - folder of ns-3 \footnote{\url{https://gitlab.com/nsnam/ns-3-dev/-/tree/ns-3.37} Version: 3.37}, where I found the following information about the simulator.
Ns-3 is a discrete-event network simulator project which was founded in 2006.
The ns-3 project is open source with a licence based on GNU GPLv2 compatibility.
It aims to provide an open, extensible network simulator for research and educational use.
Ns-3 scripts can be written in C++ or Python.

The concept of ns3 is based on the abstraction of simulated systems.
For this purpose, the term node was introduced for basic computing devices.
The Node class offers the possibility of installing protocol stacks and applications or adding peripheral cards and mobility models to the node.
Applications are the abstraction of the user-level applications, representing an activity to be simulated.
For this purpose, the applications use resources and functionalities provided by the system software of a node.

Every node gets network access via the Net Device class.
The Net Device class represents the physical interface of a node,
which can be a network interface card or peripheral card.
The Net Device simulates the software driver and the network interface hardware.

Every Net Device is connected to a channel.
The channel class represents the physical medium which is used to transmit data.
The channel behaviour is based
on the channel model, which may include interference, propagation delay and loss.

Any packets in ns-3 can be tagged with a ns-3 Packet Tag \footnote{\url{https://www.nsnam.org/docs/release/3.36/doxygen/classns3_1_1_tag.html}},
which are designed to add additional information to the packet.
Every added ns-3 Packet Tag belongs to the packet and does not change the packet size or characteristics.

The current version ns-3.37 supports IEEE 802.11ax as a standard in infrastructure and ad-hoc mode \footnote{\url{https://www.nsnam.org/docs/models/html/wifi-design.html}}.
However, support for the 802.11ax standard is not yet complete.
It is already possible to configure \ac{DCM} and \ac{STBC},
but there is a comment in line 496 of the file \textit{he-ppdu.cc} that these are
not yet taken into account in the current ns-3 version 3.37 \footnote{\url{https://www.nsnam.org/docs/models/html/wifi-user.html}}.

When examining the implementation of 802.11ax in ns-3, one notices that the implementation of 802.11ax
in ns-3 already implements a \ac{HE} \ac{ER} SU \ac{PPDU} preamble, but this is never used,
and one cannot activate the \ac{ER} mode.

Therefore, some open points of 802.11ax in ns-3 still need to be implemented.

\textcite{black_netsimulyzer_2021} have developed a 3D visualisation of ns-3, which visualises the simulations in 3D to make the ns-3 simulation
scenarios tangible.
The authors' graphical extension consists of two open source programmes.
The NetSimulyzer ns-3 module \footnote{\url{https://github.com/usnistgov/NetSimulyzer-ns3-module}} can be integrated into the ns-3 simulation and builds a JSON file using the specified functions and configurations.
This file contains all the data required for visualisation in the application NetSimulyzer \footnote{\url{https://github.com/usnistgov/NetSimulyzer}}.

